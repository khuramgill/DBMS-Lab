\documentclass[10pt]{article}

 
\usepackage[margin=1in]{geometry} 
\usepackage{amsmath,amsthm,amssymb, graphicx, multicol, array}
 
\newcommand{\N}{\mathbb{N}}
\newcommand{\Z}{\mathbb{Z}}
 
\newenvironment{problem}[2][Problem]{\begin{trivlist}
\item[\hskip \labelsep {\bfseries #1}\hskip \labelsep {\bfseries #2.}]}{\end{trivlist}}

\begin{document}
 
\title{CS262- Problem Set 1}
\author{
CS262- Database Systems\\
RegNo 2022-CS-48 | Khuram Iqbal\\}
\maketitle
\noindent
Consider the following schema.\\
Company(\underline{name}, city)\\
\textbf{Description}
Relation list the company name and location of company in city attribute. \\ \\
\noindent
Product(\underline{name}, maker, cost, year)\\ 
\textbf{Description}
Each product has name, and manufacturer of product in maker, cost as purchase price, and year as the launch year of that particular product. product name is unique for all problems except problem No.4 \\ \\
Purchase(\underline{id}, product, buyer, price)\\
\textbf{Description}
Relation list the purchases made by customer listed in buyer columns, price as sale price, and product as name of product. 
\\ \\
\noindent
\textbf{To-Do} For each of the problems given below you are required to provide Relational algebra expression and at least five equivalent solutions in SQL, out of which one solution should be performed using
\begin{enumerate}
    \item Cartesian product
    \item Joins
    \item Subquery
\end{enumerate}

If any of the above solutions is not possible provide the reason as well.\\ \\

\usepackage[margin=1in]{geometry}
\begin{problem}{1}
Find the products(names only) whose cost is more than the average cost.
\end{problem}

\begin{proof}[Solution]

\textbf{Relational Algebra:}
\[
\pi_{\text{name}}\left(\sigma_{\text{cost} > \text{avg\_cost}}(\text{Product} \bowtie_{\text{name} = \text{product}} (\pi_{\text{product}, \text{avg\_cost}}(\gamma_{\text{avg\_cost}}(\text{Product}))))
\right)
\]

\textbf{SQL Solutions:}
\begin{enumerate}
    \item Using Cartesian Product:
    \begin{verbatim}
    SELECT DISTINCT p.name FROM Product p, (SELECT AVG(cost) as     avg_cost FROM Product)
    avg_p WHERE p.cost > avg_p.avg_cost;  
    \end{verbatim}

    
    \item Using Joins:
    \begin{verbatim}
    SELECT DISTINCT p.name
    FROM Product p
    JOIN (SELECT AVG(cost) as avg_cost FROM Product) avg_p
    ON p.cost > avg_p.avg_cost;
    \end{verbatim}
    
    \item Using Subquery:
    \begin{verbatim}
    SELECT DISTINCT name
    FROM Product
    WHERE cost > (SELECT AVG(cost) FROM Product);
    \end{verbatim}

    \item Solution num: 4
    \begin{verbatim}
        SELECT DISTINCT p1.name
        FROM Product p1
        JOIN Product p2 ON p1.name = p2.name AND p2.cost > (SELECT AVG(cost) FROM Product);

    \end{verbatim}

    \item Solution num: 5
    \begin{verbatim}
        SELECT DISTINCT p1.name FROM Product p1 WHERE p1.name IN
        (SELECT name FROM Product WHERE cost > (SELECT AVG(cost) FROM Product)
    \end{verbatim}
\end{enumerate}
\end{proof}





\begin{problem}{2}
List the name of companies whose products are bought by Aslam.
\end{problem}

\begin{proof}[Solution]
\textbf{Relational Algebra:}
\[
\pi_{\text{name}}\left(\sigma_{\text{buyer} = \text{'Aslam'}}(\text{Purchase} \bowtie_{\text{product} = \text{name}} \text{Product})\right)
\]

\textbf{SQL Solutions:}
\begin{enumerate}

    \item Using WHERE Conditions:
    \begin{verbatim}
    SELECT DISTINCT Company.name
    FROM Company, Product, Purchase
    WHERE Company.name = Product.maker
      AND Product.name = Purchase.product
      AND Purchase.buyer = 'Aslam';
    \end{verbatim}
    \item Using Cartesian Product:
    \begin{verbatim}
    SELECT DISTINCT c.name
    FROM Company c, Product p, Purchase pu
    WHERE p.name = pu.product
      AND c.name = p.maker
      AND pu.buyer = 'Aslam';
    \end{verbatim}
    
    \item Using Joins:
    \begin{verbatim}
    SELECT DISTINCT c.name
    FROM Company c
    JOIN Product p ON c.name = p.maker
    JOIN Purchase pu ON p.name = pu.product
    WHERE pu.buyer = 'Aslam';
    \end{verbatim}
    
    \item Using Subquery:
    \begin{verbatim}
    SELECT DISTINCT c.name
    FROM Company c
    WHERE c.name IN (SELECT p.maker
                     FROM Product p
                     JOIN Purchase pu ON p.name = pu.product
                     WHERE pu.buyer = 'Aslam');
    \end{verbatim}
    \usepackage[margin=1in]{geometry}
    \item Solution: 5
    \begin{verbatim}
    SELECT DISTINCT Company.name
    FROM Company, Product, Purchase
    WHERE Company.name = Product.maker
      AND Product.name = Purchase.product
      AND Purchase.buyer = 'Aslam';
    \end{verbatim}
\end{enumerate}
\end{proof}


\begin{problem}{3}
List the name of products that are more expensive that all the products produced by Unilever.
\end{problem}

\begin{proof}[Solution]
\textbf{Relational Algebra:}
\[
\pi_{\text{name}}\left(\text{Product} - \pi_{\text{name}}\left(\sigma_{\text{maker} = \text{'Unilever'}}(\text{Product})\right)\right)
\]

\textbf{SQL Solutions:}
\begin{enumerate}
    \item Using Cartesian Product:
    \begin{verbatim}
    SELECT DISTINCT p1.name
    FROM Product p1, Product p2
    WHERE p1.cost > p2.cost
      AND p2.maker = 'Unilever';
    \end{verbatim}
    
    \item Using Joins:
    \begin{verbatim}
    SELECT DISTINCT p1.name
    FROM Product p1
    JOIN Product p2 ON p1.cost > p2.cost
    WHERE p2.maker = 'Unilever';
    \end{verbatim}

    \item Using Join and Subquery:
    \begin{verbatim}
    SELECT DISTINCT Product.name
    FROM Product
    WHERE Product.cost > ALL (
        SELECT cost
        FROM Product, Company
        WHERE Product.maker = Company.name
          AND Company.name = 'Unilever'
    );
    \end{verbatim}
    
    \item Using Max nad Join:
    \begin{verbatim}
    SELECT DISTINCT Product.name
    FROM Product, Company
    WHERE Product.maker = Company.name
      AND Company.name = 'Unilever'
      AND Product.cost > (
          SELECT MAX(cost)
          FROM Product, Company
          WHERE Product.maker = Company.name
            AND Company.name = 'Unilever'
      );
    \end{verbatim}
    
    \item Using Subquery:
    \begin{verbatim}
    SELECT DISTINCT name
    FROM Product p1
    WHERE cost > ALL (SELECT cost
                     FROM Product p2
                     WHERE p2.maker = 'Unilever');
    \end{verbatim}
\end{enumerate}
\end{proof}


\begin{problem}{4}
List the copy cat products along with manufacturer, i.e. the products that have the same name as produced by Unilever.
\end{problem}

\begin{proof}[Solution]
\textbf{Relational Algebra:}
\[
\pi_{\text{Product.name, maker}}\left(\sigma_{\text{Product.name} = \text{UnileverProducts.name}}\left(\text{Product} \bowtie \rho_{\text{UnileverProducts}(\text{name}, \text{'Unilever'})}(\text{Product})\right)\right)
\]

\textbf{SQL Solutions:}
\begin{enumerate}
    \item Using Cartesian Product:
    \begin{verbatim}
    SELECT DISTINCT p1.name, p1.maker
    FROM Product p1, Product p2
    WHERE p1.name = p2.name
      AND p2.maker = 'Unilever';
    \end{verbatim}
    
    \item Using Joins:
    \begin{verbatim}
    SELECT DISTINCT p1.name, p1.maker
    FROM Product p1
    JOIN Product p2 ON p1.name = p2.name
    WHERE p2.maker = 'Unilever';
    \end{verbatim}
    
    \item Using Subquery:
    \begin{verbatim}
    SELECT DISTINCT name, maker
    FROM Product
    WHERE name IN (SELECT name FROM Product WHERE maker = 'Unilever');
    \end{verbatim}
     \item Using Simple Join:
    \begin{verbatim}
    SELECT Product.name, Product.maker
    FROM Product, Company
    WHERE Product.maker = Company.name
      AND Company.name = 'Unilever'
      AND Product.name IN (
          SELECT Product.name
          FROM Product, Company
          WHERE Product.maker = Company.name
            AND Company.name <> 'Unilever'
      );
    \end{verbatim}
    
    \item Using Join with NOT EXISTS:
    \begin{verbatim}
    SELECT Product.name, Product.maker
    FROM Product, Company
    WHERE Product.maker = Company.name
      AND Company.name = 'Unilever'
      AND NOT EXISTS (
          SELECT *
          FROM Product, Company
          WHERE Product.maker = Company.name
            AND Company.name <> 'Unilever'
            AND Product.name = Product.name
      );
    \end{verbatim}
\end{enumerate}
\end{proof}


\begin{problem}{5}
Buyers of products produced in Lahore. 
\end{problem}

\begin{proof}[Solution]
textbf{Relational Algebra:}
\[
\pi_{\text{Purchase.buyer}}\left(\sigma_{\text{Product.maker} = \text{'Lahore'}}\left(\text{Product} \bowtie \text{Purchase}\right)\right)
\]

\textbf{SQL Solutions:}
\begin{enumerate}
    \item Using Cartesian Product:
    \begin{verbatim}
    SELECT DISTINCT pu.buyer
    FROM Product pr, Purchase pu
    WHERE pr.maker = 'Lahore'
      AND pr.name = pu.product;
    \end{verbatim}
    
    \item Using Joins:
    \begin{verbatim}
    SELECT DISTINCT pu.buyer
    FROM Product pr
    JOIN Purchase pu ON pr.name = pu.product
    WHERE pr.maker = 'Lahore';
    \end{verbatim}
    \item Using Simple Join:
    \begin{verbatim}
    SELECT DISTINCT Purchase.buyer
    FROM Purchase, Product, Company
    WHERE Purchase.product = Product.name
      AND Product.maker = Company.name
      AND Company.city = 'Lahore';
    \end{verbatim}
    
    \item Using Simple Join with IN:
    \begin{verbatim}
    SELECT DISTINCT buyer
    FROM Purchase
    WHERE product IN (
          SELECT Product.name
          FROM Product, Company
          WHERE Product.maker = Company.name
            AND Company.city = 'Lahore'
      );
    \end{verbatim}
    
    \item Using Subquery:
    \begin{verbatim}
    SELECT DISTINCT buyer
    FROM Purchase
    WHERE product IN (SELECT name FROM Product WHERE maker = 'Lahore');
    \end{verbatim}
\end{enumerate}
\end{proof}




\begin{problem}{6}
List of buyers, who only buy the products 'Made in Karachi'.
\end{problem}

\begin{proof}[Solution]
\textbf{Relational Algebra:}
\[
\pi_{\text{buyer}}\left(\sigma_{\text{product\_city} = 'Karachi' \land \text{count\_distinct\_products} = 1}(\text{Purchase} \bowtie_{\text{product} = \text{name}} (\gamma_{\text{count\_distinct\_products}}(\text{Product})))
\right)
\]

\textbf{SQL Solutions:}
\begin{enumerate}
    \item Using Cartesian Product:
    \begin{verbatim}
    SELECT DISTINCT pu.buyer
    FROM Purchase pu, (SELECT product, COUNT(DISTINCT name) as count_distinct_products
                      FROM Product WHERE maker = 'Karachi' GROUP BY product) prod_khi
    WHERE pu.product = prod_khi.product AND prod_khi.count_distinct_products = 1;
    \end{verbatim}
    
    \item Using Joins:
    \begin{verbatim}
    SELECT DISTINCT pu.buyer
    FROM Purchase pu
    JOIN (SELECT product, COUNT(DISTINCT name) as count_distinct_products
          FROM Product WHERE maker = 'Karachi' GROUP BY product) prod_khi
    ON pu.product = prod_khi.product AND prod_khi.count_distinct_products = 1;
    \end{verbatim}

    \item Using Simple Join:
    \begin{verbatim}
    SELECT DISTINCT P1.buyer
    FROM Purchase P1
    WHERE P1.product IN (
          SELECT DISTINCT Product.name
          FROM Product, Company
          WHERE Product.maker = Company.name
            AND Company.city = 'Karachi'
      )
      AND NOT EXISTS (
          SELECT P2.buyer
          FROM Purchase P2
          WHERE P2.buyer = P1.buyer
            AND P2.product NOT IN (
                SELECT Product.name
                FROM Product, Company
                WHERE Product.maker = Company.name
                  AND Company.city = 'Karachi'
            )
      );
    \end{verbatim}
    
    \item Using NOT IN:
    \begin{verbatim}
    SELECT DISTINCT buyer
    FROM Purchase
    WHERE buyer NOT IN (
          SELECT DISTINCT P.buyer
          FROM Purchase P, Product, Company
          WHERE P.product = Product.name
            AND Product.maker = Company.name
            AND Company.city = 'Karachi'
            AND P.product NOT IN (
                SELECT DISTINCT Product.name
                FROM Product, Company
                WHERE Product.maker = Company.name
                  AND Company.city != 'Karachi'
            )
      );
    \end{verbatim}
    
    \item Using Subquery:
    \begin{verbatim}
    SELECT DISTINCT buyer
    FROM Purchase
    WHERE product IN (SELECT product
                      FROM Product
                      WHERE maker = 'Karachi'
                      GROUP BY product
                      HAVING COUNT(DISTINCT name) = 1);
    \end{verbatim}
\end{enumerate}
\end{proof}


\begin{problem}{7}
Name and price of products bought by more than five customers.
\end{problem}

\begin{proof}[Solution]
\textbf{Relational Algebra:}
\[
\pi_{\text{name, price}}\left(\sigma_{\text{count\_distinct\_buyers} > 5}(\text{Purchase} \bowtie_{\text{product} = \text{name}} (\gamma_{\text{count\_distinct\_buyers}}(\text{Purchase})))
\right)
\]

\textbf{SQL Solutions:}
\begin{enumerate}
    \item Using Cartesian Product:
    \begin{verbatim}
    SELECT DISTINCT pr.name, pr.price
    FROM Purchase pu, (SELECT product, COUNT(DISTINCT buyer) as count_distinct_buyers
                      FROM Purchase GROUP BY product) pu_count
    JOIN Product pr ON pr.name = pu_count.product
    WHERE pu.product = pu_count.product AND pu_count.count_distinct_buyers > 5;
    \end{verbatim}

    \item Using COUNT and GROUP BY:
    \begin{verbatim}
    SELECT Product.name, Purchase.price
    FROM Product
    JOIN Purchase ON Product.name = Purchase.product
    WHERE Purchase.product IN (
          SELECT P.product
          FROM Purchase P
          GROUP BY P.product
          HAVING COUNT(DISTINCT P.buyer) > 5
      );
    \end{verbatim}
    
    \item Using Subquery:
    \begin{verbatim}
    SELECT name, price
    FROM Product
    WHERE name IN (
          SELECT P.product
          FROM Purchase P
          GROUP BY P.product
          HAVING COUNT(DISTINCT P.buyer) > 5
      );
    \end{verbatim}
    
    \item Using Joins:
    \begin{verbatim}
    SELECT DISTINCT pr.name, pr.price
    FROM Purchase pu
    JOIN (SELECT product, COUNT(DISTINCT buyer) as count_distinct_buyers
          FROM Purchase GROUP BY product) pu_count
    ON pu.product = pu_count.product AND pu_count.count_distinct_buyers > 5
    JOIN Product pr ON pr.name = pu_count.product;
    \end{verbatim}
    
    \item Using Subquery:
    \begin{verbatim}
    SELECT DISTINCT pr.name, pr.price
    FROM Product pr
    WHERE pr.name IN (SELECT pu.product
                      FROM Purchase pu
                      GROUP BY pu.product
                      HAVING COUNT(DISTINCT pu.buyer) > 5);
    \end{verbatim}
\end{enumerate}
\end{proof}


\begin{problem}{8}
List of products that are more expensive that all the products made by same company before 2015.
\end{problem}

\begin{proof}[Solution]
\textbf{Relational Algebra:}
\[
\pi_{\text{name}}\left(\sigma_{\text{price} > \text{max\_price\_before\_2015}}(\text{Product} \bowtie_{\text{name} = \text{product}} (\gamma_{\text{max\_price\_before\_2015}}(\sigma_{\text{year} < 2015}(\text{Product}))))
\right)
\]

\textbf{SQL Solutions:}
\begin{enumerate}
    \item Using Cartesian Product:
    \begin{verbatim}
    SELECT DISTINCT p.name
    FROM Product p, (SELECT product, MAX(price) as max_price_before_2015
                    FROM Product
                    WHERE year < 2015
                    GROUP BY product) max_p
    WHERE p.name = max_p.product AND p.price > max_p.max_price_before_2015;
    \end{verbatim}
    
    \item Using Joins:
    \begin{verbatim}
    SELECT DISTINCT p.name
    FROM Product p
    JOIN (SELECT product, MAX(price) as max_price_before_2015
          FROM Product
          WHERE year < 2015
          GROUP BY product) max_p
    ON p.name = max_p.product AND p.price > max_p.max_price_before_2015;
    \end{verbatim}

    \item Using NOT EXISTS:
    \begin{verbatim}
    SELECT p1.name
    FROM Product p1
    WHERE p1.cost > ALL (
          SELECT p2.cost
          FROM Product p2
          WHERE p2.maker = p1.maker AND p2.year < 2015
      );
    \end{verbatim}
    
    \item Using Subquery:
    \begin{verbatim}
    SELECT name
    FROM Product
    WHERE cost > ALL (
          SELECT cost
          FROM Product
          WHERE maker = Product.maker AND year < 2015
      );
    \end{verbatim}
    
    \item Subquery:
    \begin{verbatim}
    SELECT DISTINCT name
    FROM Product p
    WHERE price > (SELECT MAX(price)
                  FROM Product
                  WHERE year < 2015
                  GROUP BY product);
    \end{verbatim}
\end{enumerate}
\end{proof}


\begin{problem}{9}
List of companies who never sale products with loss.
\end{problem}

\begin{proof}[Solution]
\textbf{Relational Algebra:}
\[
\pi_{\text{maker}}\left(\text{Product} - \pi_{\text{maker}}\left(\sigma_{\text{price} < \text{cost}}(\text{Product})\right)\right)
\]

\textbf{SQL Solutions:}
\begin{enumerate}
    \item Using Cartesian Product:
    \begin{verbatim}
    SELECT DISTINCT p.maker
    FROM Product p
    WHERE NOT EXISTS (SELECT 1
                      FROM Product
                      WHERE maker = p.maker AND price < cost);
    \end{verbatim}
    
    \item Using Joins:
    \begin{verbatim}
    SELECT DISTINCT p.maker
    FROM Product p
    LEFT JOIN (SELECT maker, 1 as loss
               FROM Product
               WHERE price < cost) loss_p
    ON p.maker = loss_p.maker
    WHERE loss_p.maker IS NULL;
    \end{verbatim}

    \item Using NOT EXISTS:
    \begin{verbatim}
    SELECT DISTINCT maker
    FROM Product p1
    WHERE NOT EXISTS (
          SELECT *
          FROM Purchase p2
          WHERE p2.product = p1.name AND p2.price < p1.cost
      );
    \end{verbatim}

    \item Using LEFT Join:
    \begin{verbatim}
    SELECT DISTINCT p.maker
    FROM Product p
    LEFT JOIN Purchase pu ON p.name = pu.product AND pu.price < p.cost
    WHERE pu.id IS NULL;
    \end{verbatim}
    
    \item Using Subquery:
    \begin{verbatim}
    SELECT DISTINCT maker
    FROM Product
    WHERE maker NOT IN (SELECT maker
                       FROM Product
                       WHERE price < cost);
    \end{verbatim}
\end{enumerate}
\end{proof}


\begin{problem}{10}
List the products which have more than average revenue in 2015 but below average revenue in 2016
\end{problem}

\begin{proof}[Solution]
\textbf{Relational Algebra:}
\[
\pi_{\text{name}}\left(\sigma_{\text{revenue} > \text{avg\_revenue\_2015} \land \text{revenue} < \text{avg\_revenue\_2016}}
 (\text{Product} \bowtie_{\text{name} = \text{product}} (\pi_{\text{product},

\text{revenue}}(\gamma_{\text{avg\_revenue\_2015}}(\text{Product}) \bowtie_{\text{year} = 2016} \gamma_{\text{avg\_revenue\_2016}}(\text{Product}))))
\right)
\]
\begin{figure}
    \centering
    \includegraphics[width=0.5\linewidth]{WhatsApp Image 2024-02-07 at 20.31.13_e550abe6.jpg}
    \caption{Due to unstable margins}
    \label{fig1:}
\end{figure}


\textbf{SQL Solutions:}
\begin{enumerate}
    \item Using Cartesian Product:
    \begin{verbatim}
    SELECT DISTINCT p.name
    FROM Product p,
         (SELECT AVG(revenue) as avg_revenue_2015
          FROM Product
          WHERE year = 2015) avg_2015,
         (SELECT AVG(revenue) as avg_revenue_2016
          FROM Product
          WHERE year = 2016) avg_2016
    WHERE p.revenue > avg_2015.avg_revenue_2015
      AND p.revenue < avg_2016.avg_revenue_2016;
    \end{verbatim}
    
    \item Using Joins:
    \begin{verbatim}
    SELECT DISTINCT p.name
    FROM Product p
    JOIN (SELECT AVG(revenue) as avg_revenue_2015
          FROM Product
          WHERE year = 2015) avg_2015
    ON p.year = 2015
    JOIN (SELECT AVG(revenue) as avg_revenue_2016
          FROM Product
          WHERE year = 2016) avg_2016
    ON p.year = 2016
    WHERE p.revenue > avg_2015.avg_revenue_2015
      AND p.revenue < avg_2016.avg_revenue_2016;
    \end{verbatim}
    
    \item Using Subquery:
    \begin{verbatim}
    SELECT DISTINCT name
    FROM Product
    WHERE revenue > (SELECT AVG(revenue)
                     FROM Product
                     WHERE year = 2015)
      AND revenue < (SELECT AVG(revenue)
                     FROM Product
                     WHERE year = 2016);
    \end{verbatim}
\item Using FULL JOIN:
    \begin{verbatim}
    SELECT DISTINCT p.name
    FROM Product p
    FULL JOIN Purchase pu ON p.name = pu.product
    WHERE pu.price > (SELECT AVG(pu1.price) FROM Purchase pu1 WHERE YEAR(pu1.date) = 2015)
    AND pu.price < (SELECT AVG(pu2.price) FROM Purchase pu2 WHERE YEAR(pu2.date) = 2016);
    \end{verbatim}
    
\end{enumerate}
\end{proof}

\end{document}